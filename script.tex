\documentclass[handout, aspectratio=169]{beamer}
\usepackage[orientation=landscape,size=custom,width=16,height=9,scale=0.4,debug]{beamerposter} 

\input{materials/commands_pkg}
\input{materials/commands_math}
\input{materials/commands_theme}

\graphicspath{{materials/figs}}
\usepackage{relsize}
\usepackage{inputenc}
\usepackage{graphicx}
\usepackage{amsmath,amssymb}
\usepackage{booktabs}
\usepackage{tcolorbox}
\usepackage[dvipsnames]{xcolor}

\mode<presentation>
{
   \usetheme{Berlin}
    \usecolortheme{seahorse}
    \setbeamertemplate{itemize item}    
    {\color{darkgray}$\blacktriangleright$}
    \setbeamertemplate{itemize subitem}{\color{lightgray}$\blacksquare$}
}

\newcommand{\redIMP}[1]{\color{NordOrange} #1}
\newcommand{\blueIMP}[1]{\color{NordMagenta} #1}

\definecolor{greentitle}{RGB}{126,169,105}
\definecolor{redtitle}{RGB}{26,69,15}
\definecolor{bluetitle}{RGB}{30,30,100}
\definecolor{title1}{RGB}{50,50,50}

\begin{document}

\begin{frame}
    \textbf{IFAC IAV 2025 - Presentation Script}
    
    Written by: K. Anurag

    \textit{kmranrg@unm.edu}
\end{frame}

\begin{frame}
    \textbf{Slide 01 - Title:}
    
    Good Afternoon, everyone. I am Kumar Anurag from the University of New Mexico, and today I'm excited to present our work on resilient tracking in autonomous vehicles using a framework we call DRISE - Distributionally Robust Input and State Estimation.

    This work is in collaboration with Kasra Azizi and my advisor Prof. Wenbin Wan.
\end{frame}

\begin{frame}
    \textbf{Slide 02 - Introduction, Problem \& Solution:}

    Today, autonomous vehicles are becoming more common on the road. To keep them safe, they need to know their own condition - like ho wfast they are going, what direction they're facing, or if they're sliding.

    Some of this information is hard to measure directly. For example, the \textbf{sideslip angle} - which shows how much the car is sliding sideways, which is very important for safety, but it's hard to measure with normal sensors.

    Also, low-cost sensors can give \textbf{noisy or wrong data}, especially in bad weather or tricky driving situations.

    So, what's the solution?

    We use \textbf{state estimation}. That means we combine sensor data with a math model of how the vehicle behaves. This helps us guess the hidden values that sensors can't measure directly.

    But this is not always easy. Sometimes the noise is strange, or the environment changes in unexpected ways. So, we need a smarter estimation method - one that works well even when things don't go as planned.
\end{frame}

\begin{frame}
    \textbf{Slide 03 - Baseline - Input \& State Estimation (ISE):}
    
    One popular method we often use is called \textbf{ISE} - Input and State Estimation. It doesn't just estimate the vehicle's internal state. It also tries to guess \textbf{unknown inputs} - like wind or tire slip, or uneven roads, these factors affect the vehicle in unexpected ways. So ISE improves the accuracy by estimating both what we can't see and what we didn't plan for. 

    But there is a problem.

    ISE works well only when the noise follows \textbf{perfect Gaussian distribution}. It can easily break down when we get \textbf{outliers} - like strange or faulty sensor readings.

    That's why we need something more \textbf{robust}.
\end{frame}

\begin{frame}
    \textbf{Slide 04 - Baseline - Distributionally Robust Estimation (DRE)}

    The second method is designed to work \textbf{even when we don't know the exact noise behavior}. For example, if the sensor noise is not perfectly Gaussian - DRE can still give us good estimates.

    How does it do this?

    It basically forms an \textbf{ambiguity set} - which means it doesn't trust one fixed noise model. Instead, it looks at a group of possible noise types and prepares for the \textbf{worst-case} within that group.

    So, that's how DRE deals with non-Gaussian noises.

    But DRE has big limitation too - it \textbf{does not estimate unknown inputs} like ISE does. This means if something unexpected pushes the vehicle - DRE can't tell.

    Also, DRE stil struggles when the measurements contain \textbf{outliers} - large, incorrect, values that can confuse the estimator.
\end{frame}

\end{document}