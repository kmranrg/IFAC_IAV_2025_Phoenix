\documentclass[handout, aspectratio=169]{beamer}
\usepackage[orientation=landscape,size=custom,width=16,height=9,scale=0.4,debug]{beamerposter} 

\input{materials/commands_pkg}
\input{materials/commands_math}
\input{materials/commands_theme}

\graphicspath{{materials/figs}}
\usepackage{relsize}
\usepackage{inputenc}
\usepackage{graphicx}
\usepackage{amsmath,amssymb}
\usepackage{booktabs}
\usepackage{tcolorbox}
\usepackage[dvipsnames]{xcolor}

\mode<presentation>
{
   \usetheme{Berlin}
    \usecolortheme{seahorse}
    \setbeamertemplate{itemize item}    
    {\color{darkgray}$\blacktriangleright$}
    \setbeamertemplate{itemize subitem}{\color{lightgray}$\blacksquare$}
}

\newcommand{\redIMP}[1]{\color{NordOrange} #1}
\newcommand{\blueIMP}[1]{\color{NordMagenta} #1}

\definecolor{greentitle}{RGB}{126,169,105}
\definecolor{redtitle}{RGB}{26,69,15}
\definecolor{bluetitle}{RGB}{30,30,100}
\definecolor{title1}{RGB}{50,50,50}

\begin{document}

\begin{frame}
    \textbf{IFAC IAV 2025 - Presentation Script}
    
    Written by: K. Anurag
\end{frame}

\begin{frame}
    \textbf{Slide 01 - Title:}
    
    Good Afternoon, everyone. I am Kumar Anurag from the University of New Mexico, and today I'm excited to present our work on resilient tracking in autonomous vehicles using a framework we call DRISE - Distributionally Robust Input and State Estimation.

    This work is in collaboration with Kasra Azizi and my advisor Prof. Wenbin Wan.
\end{frame}

\begin{frame}
    \textbf{Slide 02 - Introduction, Problem \& Solution:}

    Today, autonomous vehicles are becoming more common on the road. To keep them safe, they need to know their own condition - like ho wfast they are going, what direction they're facing, or if they're sliding.

    Some of this information is hard to measure directly. For example, the \textbf{sideslip angle} - which shows how much the car is sliding sideways, which is very important for safety, but it's hard to measure with normal sensors.

    Also, low-cost sensors can give \textbf{noisy or wrong data}, especially in bad weather or tricky driving situations.

    So, what's the solution?

    We use \textbf{state estimation}. That means we combine sensor data with a math model of how the vehicle behaves. This helps us guess the hidden values that sensors can't measure directly.

    But this is not always easy. Sometimes the noise is strange, or the environment changes in unexpected ways. So, we need a smarter estimation method - one that works well even when things don't go as planned.
\end{frame}





\end{document}