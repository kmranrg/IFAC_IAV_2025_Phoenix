\documentclass{beamer}
\usetheme{CambridgeUS}

\title{DRISE}
\subtitle{Framework}
\author{Kumar Anurag}
\institute{University of New Mexico}
\date{May 07, 2025}

\usepackage{graphicx}
\usepackage{amsmath}
\usepackage{hyperref}


\begin{document}
\begin{frame}
Presentation Script

Author: Kumar Anurag

kmranrg@unm.edu
\end{frame}

\begin{frame}{Slide 01 - Title}
\protect\hypertarget{slide-01---title}{}
Good Afternoon, everyone. I am Kumar Anurag from the University of New
Mexico, and today I'm excited to present our work on resilient tracking
in autonomous vehicles using a framework we call \textbf{DRISE -
Distributionally Robust Input and State Estimation}.

This work is in collaboration with Kasra Azizi and my advisor
Prof.~Wenbin Wan.
\end{frame}

\begin{frame}{Slide 02 - Introduction, Problem \& Solution}
\protect\hypertarget{slide-02---introduction-problem-solution}{}
Today, autonomous vehicles are becoming more common the road. To keep
them safe, they need to know their own condition - like how fast they
are going, what direction they're facing, or if they're sliding.

Some of this information is hard to measure directly. For example, the
\textbf{sideslip angle} - which shows how much the car is sliding
sideways, which is very important for safety, but it's hard to measure
with normal sensors.

Also, low-cost sensors can give \textbf{noisy or wrong data}, especially
in bad weather or tricky driving situations.

So, what's the solution?

We use \textbf{state estimation}. That means we combine sensor data with
a math model of how the vehicle behaves. This helps us guess the hidden
values that sensors can't measure directly.
\end{frame}

\begin{frame}{Slide 02 - Introduction, Problem \& Solution (contd...)}
But this is not always easy. Sometimes the noise is strange, or the
environment changes in unexpected ways. So, we need a smarter estimation
method - one that works well even when things don't go as planned.
\end{frame}

\begin{frame}{Slide 03 - Baseline: Input and State Estimation (ISE)}
\protect\hypertarget{slide-03---baseline-input-and-state-estimation-ise}{}
One popular method we often use is called \textbf{ISE} - Input and State
Estimation. It doesn't just estimate the vehicle's internal state. It
also tries to guess \textbf{unknown inputs} - like wind or tire slip, or
uneven roads, these factors affect the vehicle in unexpected ways. So,
ISE improves the accuracy by estimating both what we can't see and what
we didn't plan for.

But there is a problem.

ISE works well only when the noise follows \textbf{perfect Gaussian
distribution}. It can easily break down when we get \textbf{outliers} -
like strange or faulty sensor readings.

That's we need something more \textbf{robust}.
\end{frame}

\begin{frame}{Slide 04 - Baseline: Distributionally Robust Estimation
(DRE)}
\protect\hypertarget{slide-04---baseline-distributionally-robust-estimation-dre}{}
The second method we looked at is \textbf{DRE} - Distributionally Robust
Estimation. It's helpful when we don't know exactly what the sensor
noise looks like. For example, if it's not a perfect Gaussian, DRE still
works.

How does it do that?

Let's look at this picture. The \textbf{black dot} in the center is the
\textbf{nominal distribution} - our best guess. But the real
distribution might be different - maybe like this \textbf{green point}.

So, DRE builds a \textbf{safe zone} around the nominal distribution,
called an \textbf{ambiguity set} - the dotted circle.

It then prepares for the \textbf{worst-case} distribution inside that
zone - the \textbf{red star}. That way, even if the real noise is
strange, our estimates are still safe and reliable.
\end{frame}

\begin{frame}{Slide 04 - Baseline: Distributionally Robust Estimation
(DRE) (contd...)}
However, DRE has some issues. It \textbf{doesn't estimate unknown
inputs}, like sudden wind or surface friction.

And it's still \textbf{sensitive to outliers} - big errors in sensor
readings.
\end{frame}

\begin{frame}{Slide 05 - Problem Formulation}
\protect\hypertarget{slide-05---problem-formulation}{}
Now, let's talk about the actual problem we are solving.

We describe the vehicle using a \textbf{linear time-varying model} -
which means the system changes over time, and we write it in terms of
matrices like \(A_k\), \(B_k\), \(C_k\).

The first equation models the \textbf{next state} of the vehicle, based
on: - the \textbf{current state} \(x_k\) - the \textbf{known control
input} \(u_k\) - like steering or acceleration - an \textbf{unknown
input} \(d_k\) - like wind, road slope, or friction that we didn't model
- and some \textbf{random noise} \(w_k\)

The second equation shows how we get our \textbf{measurements} \(y_k\) -
but again, they can include \textbf{sensor noise} \(v_k\), and even
\textbf{outliers}.

So, we have three big challenges: - there's an unknown input that
affects the system - the noise in both process and measurements is
uncertain - and we may get bad measurements - like outliers

Our goal is to \textbf{estimate the state \(x_k\) accurately}, even when
all these things happen.
\end{frame}

\begin{frame}{Slide 06 - Unknown Input Estimation}
\protect\hypertarget{slide-06---unknown-input-estimation}{}
The first challenge we tackle is the \textbf{unknown input} \(d_k\).

These are forces or effects that we didn't include in our model - like
road bumps, slippery surfaces, or even wind.

Most classical estimators (like Kalman Filter) ignore these. But in
real-world driving, they can cause \textbf{big errors} if we don't
handle them.

So, the first step in our \textbf{DRISE} algorithm is to
\textbf{estimate this unknown input}.

We do that using the formula:
\[\hat{d_{k-1}} = M_k (y_k - C_k \hat{x}_k^-)\] Let me explain it
simply: - we compare the actual measurement \(y_k\) with what we
expected \(C_k \hat{x}_k^-\) - that difference is called the
\textbf{innovation} - it tells us what's off - then we apply a gain
matrix \(M_k\) to get our best guess of the unknown input

This allows us to handle \textbf{unmodeled disturbances}, and helps make
our state estimation more accurate.
\end{frame}

\begin{frame}{Slide 06 - Unknown Input Estimation (contd...)}
\emph{Audience: How \(M_k\) is calculated? \(M_k\) is not a fixed
constant, but is computed dynamically at each time step based on system
matrix \(C_k\), state covariance \(P_k\), and measurement covariance
\(R_k\). In short, we don't choose manually, it's derived from our model
and noise statistics during prediction step of the filter.}
\end{frame}

\begin{frame}{Slide 07 - Distributionally Robust Estimation}
\protect\hypertarget{slide-07---distributionally-robust-estimation}{}
The second challenge we solve is about \textbf{noise uncertainty}.

In real life, the noise in our system and sensors may \textbf{not follow
the exact Gaussian shape} we assume. It might change over time, or act
strangely in some conditions.

If we use a normal Kalman Filter or even ISE, these assumptions can lead
to \textbf{bad estimates}.

To fix this, we use the idea of \textbf{Distributionally Robust
Estimation} or DRE.

Here's the key idea: instead of assuming one fixed noise distribution,
we prepare for \textbf{a whole set of possible noise types}. We call
this \textbf{ambiguity set}, written as \(\mathcal{F}\).

Mathematically, it looks like this:
\[\min_{estimator} \max_{P \in \mathcal{F}} \mathbb{E}_P[\text{Error}]\]
It means: Choose the estimator that gives the best performance - even if
the true noise is the worst one inside our safe zone.
\end{frame}

\begin{frame}{Slide 07 - Distributionally Robust Estimation (contd...)}
This makes the estimation more \textbf{reliable}, even when our noise
model isn't perfect.
\end{frame}

\begin{frame}{Slide 08 - Robust Update}
\protect\hypertarget{slide-08---robust-update}{}
The final challenge we deal with is \textbf{outliers} in the sensor
data.

Sometimes a sensor might give a strange reading - maybe from a glitch,
reflection, or communication delay.

These \textbf{bad values} can make the estimator jump or drift away from
the true state.

To fix this, we use a concept from \textbf{robust statistics} called the
\textbf{influence function}.

The goal is to \textbf{limit how much each measurement can affect the
estimate} - especially when that measurement looks suspicious.

We use a special function called a \textbf{Huber-type clipping
function}, written as \(\psi(\mu)\). It behaves like this: - if the
measurement difference \(\mu\) is small, we use it normally. - but if
\(\mu\) is too big - meaning it might be an outlier - we cap it at a
limit \(K\)

So large deviations don't completely break the estimate. This makes the
update step much \textbf{more robust}, especially in tough real-world
conditions.
\end{frame}

\begin{frame}{Slide 08 - Robust Update (contd...)}
\emph{Audience: How \(K\) is evaluated? The value of \(K\) is tuned
empirically. Generally, we set value of \(K\) to \(1.5\) to \(2\) times
the standard deviation of the innovation (\(\mu\) - measurement
residual).}
\end{frame}

\begin{frame}{Slide 09 - DRISE Framework Block Diagram}
\protect\hypertarget{slide-09---drise-framework-block-diagram}{}
Now that we've seen the three core building blocks, let's look at how
they all come together in the full \textbf{DRISE framework}.

On the left, we see \textbf{CARLA}, the simulator we use to model the
autonomous vehicle. It gets: - control inputs \(u_k\) from the
controller - affects from unknown inputs \(d_k\) - and disturbances from
process noise \(w_k\)

The vehicle sends back sensor measurements \(y_k\), which are also
affected by measurement noise \(v_k\).

On the right side is our \textbf{DRISE estimator} - it takes in the
control input \(u_k\) and the measurements \(y_k\), and processes them
in 4 steps: 1. Prediction: Predict where the vehicle should be, based on
the last state and control input 2. Unknown Input Estimation: Use sensor
feedback to estimate unmodeled forces \(d_k\) 3. Time Update: Refine the
state using the predicted unknown input 4. Robust Measurement Update:
Using a clipped influence function to finalize the estimate
\end{frame}

\begin{frame}{Slide 10 - The DRISE Algorithm Cycle}
\protect\hypertarget{slide-10---the-drise-algorithm-cycle}{}
\begin{enumerate}
\item Prediction: First, we predict the next state based on the last estimated state \(x_{k-1}\), and the known control input \(u_{k-1}\). This gives us a rough idea of where the vehicle should be.
\item Unknown Input Estimation: Next, we estimate the unknown input \(d_{k-1}\) - for example, disturbances like wind or road slope. We do this by comparing what the sensors saw with what we expected, and using the gain matrix \(M_k\).
\item Time Update: We refine the predicted state using the estimated unknown input. This helps us correct the state before even using the new measurement.
\item Robust Measurement Update: Finally, we apply a robust correction using the actual measurement. If the measurement seems reliable, we update normally. But if it looks like an outlier, we limit its effect using a clipped influence function \(\psi(\cdot)\).
\end{enumerate}
\end{frame}

\begin{frame}{Slide 11 - Simulation Setup}
\protect\hypertarget{slide-11---simulation-setup}{}
Now, let's talk about the simulation setup we used to test the DRISE
framework.

We use a standard \textbf{kinematic bicycle model} - it's a simplified
way to simulate how a car moves.

This includes states like \textbf{position, velocity, and yaw angle}.

The \textbf{inputs} to the system are: - Steering angle - Acceleration

We also introduce an \textbf{unknown input} - a time-varying signal,
which mimics things like road slope or friction changes that we don't
model directly.

For realism, we include: - Process noise covariance \(Q_k\), which adds
uncertainty in how the system evolves. - Measurement noise covariance
\(R_k\), which affects the sensor readings.

And we make the test more challenging by injecting \textbf{outliers} and
\textbf{noise deviations}.
\end{frame}

\begin{frame}{Slide 11 - Simulation Setup (contd...)}
Finally, we compare our method - \textbf{DRISE} - with three baselines:
- Kalman Filter - Input and State Estimation - Distributionally Robust
Estimation

This setup helps us understand how well DRISE performs under tough and
realistic conditions.
\end{frame}

\begin{frame}{Slide 12 - CARLA Simulation Environment}
\protect\hypertarget{slide-12---carla-simulation-environment}{}
To test how well DRISE works in real-world-like conditions, we used
CARLA simulator.

CARLA is a powerful \textbf{open source platform} made for autonomous
vehicle research.

It lets us simulate driving in \textbf{realistic urban environments},
with buildings, traffic, trees and even road textures.

It also provides detailed models for \textbf{sensors and vehicle
dynamics}, so we can see how estimation performs under realistic
conditions.

Most importantly, CARLA allows us to introduce \textbf{uncertainties}
like unexpected forces.

So, this setup gives us a \textbf{challenging and believable}
environment for evaluating our framework.
\end{frame}

\begin{frame}{Slide 13, 14, and 15 - Results}
\protect\hypertarget{slide-13-14-and-15---results}{}
First, this plot shows the \textbf{state estimation error} - how far off
the estimators are from the true vehicle state. Here, we can see that
the Kalman Filter performs the worst. It diverges when the noise or
inputs don't match the assumptions. But DRISE clearly outperforms all,
with the lowest overall error.

In the second plot, we show how well each method estimates
\textbf{unknown inputs} - like wind, slope, or forces. \textbf{DRISE
again shows much better accuracy}.

In the third plot, we show how well the \textbf{vehicle follows a
planned 2D path}. \textbf{DRISE produces the closest math to the
reference trajectory}.

Conclusion: This shows combining robustness and unknown input-state
estimation really improves accuracy.
\end{frame}

\begin{frame}{Slide 16 - Future Work}
\protect\hypertarget{slide-16---future-work}{}
In future work, we also plan to explore different ways of defining
ambiguity set - specifically using Wasserstein and KL divergence-based
methods.
\end{frame}

\begin{frame}{Slide 17 - Thank You}
\protect\hypertarget{slide-17---thank-you}{}
This brings me to the end of my presentation. I'd be happy to take
questions now.
\end{frame}

\end{document}